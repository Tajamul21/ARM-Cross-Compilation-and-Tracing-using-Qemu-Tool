\documentclass[a4paper,12pt]{article}
\usepackage{graphicx}
%\usepackage{url}
%\usepackage{hyperref}
%\hypersetup{
%colorlinks=true,
%linkcolor=blue,
%urlcolor=blue,}
\title {COMPUTER ORGANIZATION AND ARCHITECTURE\\   Assignment 1 }
\author{by \\Tajamul Ashraf \\ And \\Fuzayil-Bin-Afzal Mir}
\date{\today}

\begin{document}	
\maketitle
\newpage
\tableofcontents
\newpage

\section{ARM cross compiler}
\subsection{Introduction}
\paragraph{ARM cross compiler}
Cross-compilation is the process of compiling code for one computer system (also known as the target) on a different system, called the host.  
A  cross compiler is a type of compiler, that generates machine code targeted to run on a system different than the one genrating it.Like we used arm cross compiler to convert from X86 processsor to ARM.
The Process of creating executable code for different machines is called re-targeting.The cross compiler is also called as retargetable compiler.We have usesd GNU GCC cross compiler.
A cross compiler is a compiler capable of creating executable code for a platform other than the one on which the compiler is running. For example, a compiler that runs on a
X86 processor but generates code that runs on ARM Processor is a cross compiler.

A cross compiler is necessary to compile code for multiple platforms from one development host. Direct compilation on the target platform might be infeasible, for example on X86 using .o file
, because those systems contain no operating system. In generalization, one computer runs multiple operating systems and a cross compiler could generate an executable for each of them from one main source.

Cross compilers are distinct from source-to-source compilers. A cross compiler is for cross-platform software development of machine 
code, while a source-to-source compiler translates from one programming language to another in text code. Both are programming tools.  
The GNU Arm Embedded Toolchain is a ready-to-use, open-source suite of tools for C(As in our case), C++ and assembly programming.
The GNU Arm Embedded Toolchain targets the 32-bit Arm Cortex-A, Arm Cortex-M, and Arm Cortex-R processor families.
\section{Ubuntu}
\subsection{How to install ubuntu}
1.Open the Ubuntu website. Go to %\url{https://www.ubuntu.com/download/desktop}

You can download the Ubuntu disk image (also known as an ISO file) here.\\
2.Install VirtualBox.\\
3.Select your Ubuntu ISO.
Go to the folder into which the Ubuntu ISO file downloaded (e.g., Desktop), then click the ISO file to select it.\\
4.Check the "Erase disk and install Ubuntu" box.\\
5.Personalize the ubuntu.\\
6.Restart the virtual machine. Once you see the Restart Now button,
do the following:
click the Exit button in the upper-right corner of the window (Windows) or the upper-left corner of the window (Mac), 
check the "Power off the machine" box, click OK.\\
7.You are ready to use ubuntu.
\subsection{How to Install arm-linux-gnueabi cross compiler}
1.Visit the link :$ https://www.acmesystems.it/arm9_toolchain$\\
2.As I am using an Arietta, open the terminal of ubuntu(by crtl+alt+t)\\
write:\\
$sudo apt-get install gcc-arm-linux-gnueabi g++-arm-linux-gnueabi$
3.After writing the above code, package will start downloading.\\
4.You are ready to use ARM cross compiler

\section{Qemu tool  }
\subsection{How to install Qemu tool on your system.}
1.Open th Terminal in Ubuntu\\
2.Run update command to update package repositories and get latest package information.\\
 $sudo apt-get update -y$\\
3.Run the install command with -y flag to quickly install the packages and dependencies.\\
 $sudo apt-get install -y qemu-user-static$\\
4.Check the system logs to confirm that there are no related errors.
\subsection{How to run the generated executable using qemu tool.} 
1.First form the .c file.\\
2.create a .s file.\\
3.We have to run the binary with Qemu\\
write :$ qemu-arm -L/usr/arm-linux-gnueabi/ program name.o$\\
\section{Program Hello World}
1.Open the Terminal.\\
 \begin{figure}[h]
	\includegraphics[width=0.2\textwidth]{1.jpg}
	\caption{ 4.1}
\end{figure}
2.Write the command: touch  helloworld.c \\
\begin{figure}[h]
	\includegraphics[width=0.2\textwidth]{2.jpg}
	\caption{ 4.2}
\end{figure}
3.Open the file and write the code for helloworld as shown in fig 4.1 and 4.2\\
4.Write the command: gcc helloworld.c -o helloworld\\
5.Your helloworld.c file is ready\\

\subsection{ How to generate the assembly code for the hello-world.c program using the crosscompiler. }
Now for the assembly code using cross compiler\\
 \begin{figure}[h]
	\includegraphics[width=0.2\textwidth]{3.jpg}
	\caption{ 4.3}
\end{figure}
\begin{figure}[h]
	\includegraphics[width=0.2\textwidth]{4.jpg}
	\caption{ 4.4}
\end{figure}
1.Write the command gcc -sS helloworld.c \\
Ls\\
cat helloworld.s\\
2. Your helloworld.s file (assembly code)is ready
\newpage
\subsection{How to generate the executable of the program using the cross-compiler. }
In the Terminal 
use the follwing commands: \\
\begin{figure}[h]
	\includegraphics[width=0.2\textwidth]{5.jpg}
	\caption{ 4.5}
\end{figure}
arm-linux-gnueabi-gcc helloworld.c -o helloworld.o\\
qemu-arm -L /usr/arm-linux-gmueabi/ helloworld.o
\subsection{Program Tracing }
	\begin{tabular}{|1|1|1|1|1|1|1|}
		\begin{tabular}{|1|1|1|1|1|1|1|}
			\hline
			Instruction& DESTIN & ATION & SR & C1 & SR & C2 \\ 
			\hline
			\hline
			\hline
			& loc & value & loc & value & loc & value \\ 
			\hline
			str r3,[fp,#-16]& address & 84 & r3 & 4 & fp & 100 \\
			\hline
			ldr r3,[fp,#-20]& r3 & 6 & address & 80 & fp & 100 \\ 
			\hline
			add r3,r2,r3& r3 & 10 & r2 & 4 & r3 & 6 \\ 
			\hline
			ldr r0,. L2 & r0 & value & loc & value &  &  \\ 
			\hline
			bl printf & lr & line 33 & func & printf &  &  \\ 
			\hline
			
	\end{tabular}
\section{Program Add}
1.Open the Terminal.\\
\newpage
\begin{figure}[h]
	\includegraphics[width=0.2\textwidth]{6.jpg}
	\caption{ 5.6}
\end{figure}
2.Write the command: touch  add.c \\

\begin{figure}[h]
	\includegraphics[width=0.2\textwidth]{7.jpg}
	\caption{ 5.7}
\end{figure}
3.Open the file and write the code for program add as shown in fig 5.6 and 5.7\\
4.Write the command: gcc add.c -o add\\
5.Your add.c file is ready\\

\subsection{ How to generate the assembly code for the add.c program using the crosscompiler. }
Now for the assembly code using cross compiler\\
\begin{figure}[h]
	\includegraphics[width=0.2\textwidth]{8.jpg}
	\caption{ 5.8}
\end{figure}
\begin{figure}[h]
	\includegraphics[width=0.2\textwidth]{4.jpg}
	\caption{ 5.9}
\end{figure}
1.Write the command gcc -sS add.c \\
Ls\\
cat add.s\\
2. Your add.s file (assembly code)is ready
\newpage
\subsection{How to generate the executable of the program using the cross-compiler. }
In the Terminal 
use the follwing commands: \\
\begin{figure}[h]
	\includegraphics[width=0.2\textwidth]{10.jpg}
	\caption{ 5.5}
\end{figure}
arm-linux-gnueabi-gcc add.c -o add.o\\
qemu-arm -L /usr/arm-linux-gmueabi/ add.o
\subsection{Program Tracing  }
\begin{tabular}{|1|1|1|1|1|1|1|}
		\hline
		Instruction& DESTIN & ATION & SR & C1 & SR & C2 \\ 
		\hline
		\hline
		\hline
		& loc & value & loc & value & loc & value \\ 
		\hline
		str r3,[fp,#-16]& address & 84 & r3 & 4 & fp & 100 \\
		\hline
		ldr r3,[fp,#-20]& r3 & 6 & address & 80 & fp & 100 \\ 
		\hline
		add r3,r2,r3& r3 & 10 & r2 & 4 & r3 & 6 \\ 
		\hline
		ldr r0,. L2 & r0 & value & loc & value &  &  \\ 
		\hline
		bl printf & lr & line 33 & func & printf &  &  \\ 
		\hline
		& loc & value & loc & value & loc & value \\ 
		\hline
		str r3,[fp,#-16]& address & 84 & r3 & 4 & fp & 100 \\
		\hline
		ldr r3,[fp,#-20]& r3 & 6 & address & 80 & fp & 100 \\ 
		\hline
		add r3,r2,r3& r3 & 10 & r2 & 4 & r3 & 6 \\ 
		\hline
		ldr r0,. L2 & r0 & value & loc & value &  &  \\ 
		\hline
		bl printf & lr & line 33 & func & printf &  &  \\ 
		\hline
		& loc & value & loc & value & loc & value \\ 
		\hline
		str r3,[fp,#-16]& address & 84 & r3 & 4 & fp & 100 \\
		\hline
		ldr r3,[fp,#-20]& r3 & 6 & address & 80 & fp & 100 \\ 
		\hline
		add r3,r2,r3& r3 & 10 & r2 & 4 & r3 & 6 \\ 
		\hline
		ldr r0,. L2 & r0 & value & loc & value &  &  \\ 
		\hline
		bl printf & lr & line 33 & func & printf &  &  \\ 
		\hline\hline
		add r3,r2,r3& r3 & 10 & r2 & 4 & r3 & 6 \\ 
		\hline
		ldr r0,. L2 & r0 & value & loc & value &  &  \\ 
		\hline
		bl printf & lr & line 33 & func & printf &  &  \\ 
		\hline
		
\end{tabular}

\section{Program Factorial}
1.Open the Terminal.\\
\newpage
\begin{figure}[h]
	\includegraphics[width=0.2\textwidth]{11.jpg}
	\caption{ 6.11}
\end{figure}
2.Write the command: touch  factorial.c \\

\begin{figure}[h]
	\includegraphics[width=0.2\textwidth]{12.jpg}
	\caption{ 6.12}
\end{figure}
3.Open the file and write the code for program factorial as shown in fig 6.11 and 6.12\\
4.Write the command: gcc factorial.c -o factorial\\
5.Your factorial.c file is ready\\

\subsection{ How to generate the assembly code for the Factorial.c program using the crosscompiler. }
Now for the assembly code using cross compiler\\
\begin{figure}[h]
	\includegraphics[width=0.2\textwidth]{13.jpg}
	\caption{ 6.13}
\end{figure}
\begin{figure}[h]
	\includegraphics[width=0.2\textwidth]{14.jpg}
	\caption{ 6.14}
\end{figure}
\begin{figure}[h]
	\includegraphics[width=0.2\textwidth]{15.jpg}
	\caption{ 6.15}
\end{figure}
1.Write the command gcc -sS factorial.c \\
Ls\\
cat factorial.s\\
2. Your factorial.s file (assembly code)is ready
\newpage
\subsection{How to generate the executable of the program using the cross-compiler. }
In the Terminal 
use the follwing commands: \\
\begin{figure}[h]
	\includegraphics[width=0.2\textwidth]{16.jpg}
	\caption{ 6.16}
\end{figure}
arm-linux-gnueabi-gcc factorial.c -o factorial.o\\
qemu-arm -L /usr/arm-linux-gmueabi/ factorial.o
\subsection{Program Tracing  }
\begin{tabular}{|1|1|1|1|1|1|1|}
	\hline
	Instruction& DESTIN & ATION & SR & C1 & SR & C2 \\ 
	\hline
	\hline
	\hline
	& loc & value & loc & value & loc & value \\ 
	\hline
	str r3,[fp,#-16]& address & 84 & r3 & 4 & fp & 100 \\
	\hline
	ldr r3,[fp,#-20]& r3 & 6 & address & 80 & fp & 100 \\ 
	\hline
	add r3,r2,r3& r3 & 10 & r2 & 4 & r3 & 6 \\ 
	\hline
	ldr r0,. L2 & r0 & value & loc & value &  &  \\ 
	\hline
	bl printf & lr & line 33 & func & printf &  &  \\ 
	\hline
		& loc & value & loc & value & loc & value \\ 
	\hline
	str r3,[fp,#-16]& address & 84 & r3 & 4 & fp & 100 \\
	\hline
	ldr r3,[fp,#-20]& r3 & 6 & address & 80 & fp & 100 \\ 
	\hline
	add r3,r2,r3& r3 & 10 & r2 & 4 & r3 & 6 \\ 
	\hline
	ldr r0,. L2 & r0 & value & loc & value &  &  \\ 
	\hline
	bl printf & lr & line 33 & func & printf &  &  \\ 
	\hline
		& loc & value & loc & value & loc & value \\ 
	\hline
	str r3,[fp,#-16]& address & 84 & r3 & 4 & fp & 100 \\
	\hline
	ldr r3,[fp,#-20]& r3 & 6 & address & 80 & fp & 100 \\ 
	\hline
	add r3,r2,r3& r3 & 10 & r2 & 4 & r3 & 6 \\ 
	\hline
	ldr r0,. L2 & r0 & value & loc & value &  &  \\ 
	\hline
	bl printf & lr & line 33 & func & printf &  &  \\ 
	\hline\hline
	add r3,r2,r3& r3 & 10 & r2 & 4 & r3 & 6 \\ 
	\hline
	ldr r0,. L2 & r0 & value & loc & value &  &  \\ 
	\hline
	bl printf & lr & line 33 & func & printf &  &  \\ 
	\hline
	
\end{tabular}
\end{document}